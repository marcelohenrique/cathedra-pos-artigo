%% abtex2-modelo-artigo.tex, v-1.9.1 laurocesar
%% Copyright 2012-2013 by abnTeX2 group at http://abntex2.googlecode.com/ 
%%
%% This work may be distributed and/or modified under the
%% conditions of the LaTeX Project Public License, either version 1.3
%% of this license or (at your option) any later version.
%% The latest version of this license is in
%%   http://www.latex-project.org/lppl.txt
%% and version 1.3 or later is part of all distributions of LaTeX
%% version 2005/12/01 or later.
%%
%% This work has the LPPL maintenance status `maintained'.
%% 
%% The Current Maintainer of this work is the abnTeX2 team, led
%% by Lauro César Araujo. Further information are available on 
%% http://abntex2.googlecode.com/
%%
%% This work consists of the files abntex2-modelo-artigo.tex and
%% abntex2-modelo-references.bib
%%

% ------------------------------------------------------------------------
% ------------------------------------------------------------------------
% abnTeX2: Modelo de Artigo Acadêmico em conformidade com
% ABNT NBR 6022:2003: Informação e documentação - Artigo em publicação 
% periódica científica impressa - Apresentação
% ------------------------------------------------------------------------
% ------------------------------------------------------------------------

\documentclass[
	% -- opções da classe memoir --
	article,			% indica que é um artigo acadêmico
	11pt,				% tamanho da fonte
	oneside,			% para impressão apenas no verso. Oposto a twoside
	a4paper,			% tamanho do papel. 
	% -- opções da classe abntex2 --
	%chapter=TITLE,		% títulos de capítulos convertidos em letras maiúsculas
	%section=TITLE,		% títulos de seções convertidos em letras maiúsculas
	%subsection=TITLE,	% títulos de subseções convertidos em letras maiúsculas
	%subsubsection=TITLE % títulos de subsubseções convertidos em letras maiúsculas
	% -- opções do pacote babel --
	english,			% idioma adicional para hifenização
	brazil,				% o último idioma é o principal do documento
	sumario=tradicional
	]{abntex2}


% ---
% PACOTES
% ---

% ---
% Pacotes fundamentais 
% ---
\usepackage{lmodern}			% Usa a fonte Latin Modern
\usepackage[T1]{fontenc}		% Selecao de codigos de fonte.
\usepackage[utf8]{inputenc}		% Codificacao do documento (conversão automática dos acentos)
\usepackage{indentfirst}		% Indenta o primeiro parágrafo de cada seção.
\usepackage{nomencl} 			% Lista de simbolos
\usepackage{color}				% Controle das cores
\usepackage{graphicx}			% Inclusão de gráficos
\usepackage{microtype} 			% para melhorias de justificação
% ---
		
% ---
% Pacotes adicionais, usados apenas no âmbito do Modelo Canônico do abnteX2
% ---
\usepackage{lipsum}				% para geração de dummy text
% ---
		
% ---
% Pacotes de citações
% ---
\usepackage[brazilian,hyperpageref]{backref}	 % Paginas com as citações na bibl
\usepackage[alf]{abntex2cite}	% Citações padrão ABNT

\usepackage{color}
\usepackage{multirow}
\usepackage[table]{xcolor}
% ---

% ---
% Configurações do pacote backref
% Usado sem a opção hyperpageref de backref
\renewcommand{\backrefpagesname}{Citado na(s) página(s):~}
% Texto padrão antes do número das páginas
\renewcommand{\backref}{}
% Define os textos da citação
\renewcommand*{\backrefalt}[4]{
	\ifcase #1 %
		Nenhuma citação no texto.%
	\or
		Citado na página #2.%
	\else
		Citado #1 vezes nas páginas #2.%
	\fi}%
% ---

% ---
% Informações de dados para CAPA e FOLHA DE ROSTO
% ---
\titulo{Uma Visão Ágil sobre o Processo de Desenvolvimento de Software do MPDFT}
\autor{Marcelo Henrique de Oliveira Lima}
\data{2014}
\instituicao{Pós-Graduação em Gestão de TI na Administração Pública}
\local{Brasil}
\orientador{Renato Brown}
% ---

% ---
% Configurações de aparência do PDF final

% alterando o aspecto da cor azul
\definecolor{blue}{RGB}{41,5,195}

% informações do PDF
\makeatletter
\hypersetup{
     	%pagebackref=true,
		pdftitle={\@title}, 
		pdfauthor={\@author},
    	pdfsubject={Modelo de artigo científico com abnTeX2},
	    pdfcreator={LaTeX with abnTeX2},
		pdfkeywords={abnt}{latex}{abntex}{abntex2}{atigo científico}, 
		colorlinks=true,       		% false: boxed links; true: colored links
    	linkcolor=blue,          	% color of internal links
    	citecolor=blue,        		% color of links to bibliography
    	filecolor=magenta,      		% color of file links
		urlcolor=blue,
		bookmarksdepth=4
}
\makeatother
% --- 

% ---
% compila o indice
% ---
\makeindex
% ---

% ---
% Altera as margens padrões
% ---
\setlrmarginsandblock{4cm}{4cm}{*}
\setulmarginsandblock{4cm}{4cm}{*}
\checkandfixthelayout
% ---

% --- 
% Espaçamentos entre linhas e parágrafos 
% --- 

% O tamanho do parágrafo é dado por:
\setlength{\parindent}{1.3cm}

% Controle do espaçamento entre um parágrafo e outro:
\setlength{\parskip}{0.2cm}  % tente também \onelineskip

% Espaçamento simples
\SingleSpacing

% ----
% Início do documento
% ----
\begin{document}

% Retira espaço extra obsoleto entre as frases.
\frenchspacing 

% ----------------------------------------------------------
% ELEMENTOS PRÉ-TEXTUAIS
% ----------------------------------------------------------

%---
%
% Se desejar escrever o artigo em duas colunas, descomente a linha abaixo
% e a linha com o texto ``FIM DE ARTIGO EM DUAS COLUNAS''.
% \twocolumn[    		% INICIO DE ARTIGO EM DUAS COLUNAS
%
%---
% página de titulo
\maketitle

% resumo em português
\begin{resumoumacoluna}

   \textcolor{red}{ESCREVER RESUMO!!!}

   \vspace{\onelineskip}

   \noindent
   \textbf{Palavras-chaves}: Processo Unificado. MPDFT. Métodos Ágeis.
\end{resumoumacoluna}

% ]  				% FIM DE ARTIGO EM DUAS COLUNAS
% ---

% ----------------------------------------------------------
% ELEMENTOS TEXTUAIS
% ----------------------------------------------------------
\textual

% ----------------------------------------------------------
% Introdução
% ----------------------------------------------------------
\section*{Introdução}
\addcontentsline{toc}{section}{Introdução}

Desde que Gordon Earle Moore, co-fundador e presidente da Intel Corporation,
profetizou que o número de transistores dos chips dobraria pelo mesmo custo a
cada dezoito meses vemos uma revolução tecnológica. Atualmente, quase tudo está
ligado à tecnologia, que faz o mundo se mover cada vez mais rápido. Isso não
muda para o ambiente de desenvolvimento de software. Essa área vem se
desenvolvendo e se estruturando desde meados dos anos setenta quando a
programação estruturada dava forma ao desenvolvimento de software. Com o advento
da programação orientada a objetos tudo mudou mais uma vez e as técnicas tiveram
que evoluir e se adaptar ao novo paradigma.

Com o mundo orientado a objetos surgiram processos cada vez mais robustos de
desenvolvimento de software, o que culminou no mais conhecido processo de
desenvolvimento orientado a objetos, o \textit{Rational Unified Process} (RUP).
A partir do RUP surgiu o Processo Unificado (PU) que mais tarde foi taxado de
processo pesado com o surgimento das técnicas ágeis. Os responsáveis pelo
desenvolvimento dos métodos ágeis buscavam simplicidade e eficiência. Como o
próprio manifesto ágil diz, eles estão ``descobrindo maneiras melhores de
desenvolver software'' \cite{agilemanifesto}.

Com essa nova abordagem no desenvolvimento de software, o profissionalismo, a
melhora nos processos, as avaliações de qualidade dos processos como CMM e
MPS.Br, veio a padronização nas organizações. O Governo Federal não ficou de
fora, o Tribunal de Contas da União (TCU) realizou uma ampla auditoria na Esfera
Federal e constatou que há muitas falhas na governança de TI
\cite{dou-20080818}. Com isso, o Ministério Público do Distrito Federal e
Territórios (MPDFT) foi provocado a se adequar e elaborou seu Processo de
Desenvolvimento de Software o MPDFT-UP.

O objetivo deste artigo é identificar gargalos, falhas e más práticas no
processo atual e propor melhorias para essas situações identificadas no
MPDFT-UP. Para isso, é feito um estudo sobre o modelo de processo atual e sobre
algumas novas técnicas que estão sendo introduzidas aos poucos no órgão. É feito
também um comparativo com o Processo Unificado e com metodologias ágeis e suas
práticas.

Na \autoref{processo-unificado} é detalhado o Processo Unificado (PU) que foi
derivado do Processo Unificado da Rational (\textit{Rational Unified Process -
RUP}). Na \autoref{metodos-ageis} é feita uma introdução ao universo ágil, é
explicado o que vem a ser o manifesto ágil, a metodologia de gerenciamento de
projetos ágeis Scrum e a metodologia de desenvolvimento de software ágil
\textit{eXtreme Programming} (XP). O MPDFT é introduzido rapidamente assim como
o MPDFT-UP, que é o processo de desenvolvimento utilizado no órgão, na
\autoref{mpdft}. A avaliação que é objetivo deste artigo encontra-se na
\autoref{avaliacao}, bem como um detalhamento dos problemas encontrados e
algumas sugestões de melhorias para os problemas encontrados. Por fim, são
feitas as considerações finais.

\section{Processo Unificado}

\label{processo-unificado}

Um processo de software é um conjunto de atividades que leva à produção de um
produto de software \cite{sommerville2007}. O Processo Unificado (PU)
\cite{jacobson1999unified} para desenvolvimento de software surgiu como uma
alternativa ao já reconhecidamente ineficiente modelo sequencial ou em cascata.
O modelo em cascata foi o primeiro a ser publicado e teve forte influência dos
processos mais gerais de engenharia de sistema \cite{sommerville2007}. Esse
modelo (cascata) caracteriza-se pelo encadeamento das fases e pelo fato da fase
seguinte não iniciar antes da atual ter terminado. Alguns estudos de
sucesso/falha mostram que projetos que utilizaram o modelo em cascata obtiveram
alta taxa de falha. Acredita-se que esse modelo ganhou forte adoção baseado em
boatos e crenças ao longo dos anos \cite{larman2007utilizando}.

Já o Processo Unificado é fortemente baseado em um modelo
iterativo e incremental. Esse modelo se mostrou mais eficiente em situações que os
requisitos não estão completamente definidos e a entrega de resultados deve ser
antecipada o máximo possível, além de reduzir a quantidade de defeitos.
\citeonline{larman2007utilizando} cita em seu livro que o desenvolvimento
iterativo e evolutivo considera normal que a fase de desenvolvimento comece
antes dos requisitos terem sido definidos em detalhes; a realimentação é usada
para esclarecer e aperfeiçoar as especificações em evolução.

Para fornecer um melhor entendimento sobre o Processo Unificado é comum
dividi-lo em três perspectivas \cite{sommerville2007}:

\begin{enumerate}
   \item Uma perspectiva dinâmica, que mostra as fases do modelo ao longo do
   tempo.
   \item Uma perspectiva  estática, que mostra as atividades realizadas no
   processo.
   \item Uma perspectiva prática, que sugere as boas práticas a serem usadas
   durante o processo.
\end{enumerate}

Conforme citado anteriormente, o Processo Unificado, assim como o modelo
tradicional e vários outros modelos, define um conjunto de fases a serem
seguidas. As quatro principais fases são:

\begin{description}
   \item[1. Concepção] - O objetivo da fase de concepção é estabelecer um
   \textit{business case} para o sistema.
   \item[2. Elaboração] Os objetivos da fase de elaboração são desenvolver um
   entendimento do domínio do problema, estabelecer um framework de arquitetura
   para o sistema, desenvolver o plano de projeto e identificar os riscos
   principais do projeto.
   \item[3. Construção] A fase de construção está essencialmente relacionada ao
   projeto, programação e teste de sistema.
   \item[4. Transição] A fase final do PU está relacionada à transferência do
   sistema da comunidade de desenvolvimento para a comunidade dos usuários e com
   a entrada do sistema em funcionamento no ambiente real.
\end{description}

Apesar das fases do PU se parecerem com as do modelo cascata elas não coincidem
com as atividades do processo e estão mais estritamente ligadas ao negócio do
que a assuntos técnicos \cite{sommerville2007}. Outra característica marcante
da estrutura de fases do PU é que elas podem se sobrepor ao longo do
desenvolvimento, enquanto no modelo tradicional uma fase não inicia antes da
finalização da fase atual.

Outra perspectiva, a estática, é organizada em disciplinas ou fluxos de trabalho
(\textit{workflows}). Uma disciplina é uma sequência de atividades que produz um
resultado de valor observável \cite{Corporation1998}. Por isso a disciplina é a
parte do processo responsável por uma das quatro perguntas básicas de um
processo (quem, o quê, como e quando), o quando. É a disciplina que faz com que
as outras perguntas sejam respondidas no momento adequado do ciclo de vida do
processo. O PU possui nove disciplinas núcleo, seis agrupadas com o título de
engenharia e outras três sob o título de suporte.

\textbf{Disciplinas de Engenharia}

\begin{description}
   \item[1. Modelagem de negócios] - Os processos de negócios são modelados
   usando casos de uso de negócios.
   \item[2. Requisitos] - Os agentes que interagem com o sistema são
   identificados e os casos de uso são desenvolvidos para modelar os requisitos de sistema.
   \item[3. Análise e projeto (design)] - Um modelo de projeto é criado e
   documentado usando modelos de arquitetura, modelos de componente, modelos de
   objeto e modelos de sequência.
   \item[4. Implementação] - Os componentes de sistema são implementados e
   estruturados em subsistemas de implementação.
   \item[5. Teste] - O teste é um processo iterativo realizados em conjunto com
   a implementação.
   \item[6. Implantação] - Uma versão do produto é criada, distribuída aos
   usuários e instalada no local de trabalho.
\end{description}

\textbf{Disciplinas de Apoio/Suporte}

\begin{description}
   \item[1. Gestão de Configuração e Mudança] - Disciplina responsável por
   gerenciar as mudanças do sistema.
   \item[2. Gerência de Projeto] - Disciplina que gerencia o desenvolvimento do
   sistema.
   \item[3. Ambiente] - Disciplina relacionada à disponibilização de ferramentas
   apropriadas de software para a equipe de desenvolvimento.
\end{description}

A terceira perspectiva do Processo Unificado é a prática, que descreve boas
práticas de engenharia de software recomendadas para uso em desenvolvimento de
sistemas \cite{sommerville2007}. São seis:

\begin{description}
   \item[1. Desenvolver software iterativamente] - Planejar os incrementos de
   software com base nas prioridades do cliente e desenvolver e entregar antes
   as características de sistema de maior prioridade no processo de
   desenvolvimento.
   \item[2. Gerenciar requisitos] - Documentar explicitamente os requisitos do
   cliente e manter acompanhamento das mudanças desses requisitos.
   \item[3. Usar arquiteturas baseadas em componentes] - Estruturar a
   arquitetura de sistemas com componentes.
   \item[4. Modelar o software visualmente] - Usar modelos gráficos de UML para
   apresentar as visões estática e dinâmicas do software.
   \item[5. Verificar a qualidade do software] - Garantir que o software atenda
   aos padrões de qualidade da organização.
   \item[6. Controlar as mudanças de software] - Gerenciar as mudanças de
   software, usando um sistema de gerenciamento de mudanças e procedimentos e
   ferramentas de gerenciamento de configuração.
\end{description}

A separação em perpectivas visa dar uma abordagem mais didática, mas o comum é
que elas sejam visualizadas sob uma única ótica, ou seja, na prática as
perspectivas estão sobrepostas conforme \autoref{graficodasbaleias}.


\begin{figure}[htb]
   \caption{Processo Unificado: Disciplinas vs Fases.}
   \label{graficodasbaleias}
   \begin{center}
       \includegraphics[scale=0.8]{graficodasbaleias.jpg}
   \end{center}
   \fonte{http://www.sutilab.com/cc/?aula=41}
\end{figure}

Essa visão de duas dimensões, para fases e disciplinas, foi uma grande evolução
na forma de desenvolver software, pois enquanto este visa entregar um produto de
trabalho, um artefato, aquele busca atingir um objetivo de desenvolvimento.
Apesar de parecer pouco isso foi uma mudança significativa na forma de escrever
software. O PU proporcionou também um modelo flexível e adaptável que foi base
para muitos outros.

\section{Métodos Ágeis}

\label{metodos-ageis}

Por volta da década de 1990 houve um movimento envolvendo desenvolvedores que
não concordavam com o caminho que estava sendo tomado pelos processos de
software. A burocracia, o peso do processo em detrimento do próprio produto a
ser desenvolvido, o foco em artefatos que não necessariamente agregam valor,
todos esses fatores foram motivos para a migração que levou a outro extremo, a
redução brusca da burocracia. Isso fez com que esse grupo de desenvolvedores
se reunissem e elaborassem uma séries de princípios a serem seguidos. Eles são
conhecidos com os doze princípios do software ágil \cite{agilemanifesto}.

\begin{enumerate}
   \item Nossa maior prioridade é satisfazer o cliente através da entrega
   contínua e adiantada de software com valor agregado.
   \item Mudanças nos requisitos são bem-vindas, mesmo tardiamente no
   desenvolvimento. Processos ágeis tiram vantagem das mudanças visando
   vantagem competitiva para o cliente.
   \item Entregar frequentemente software funcionando, de poucas semanas a
   poucos meses, com preferência à menor escala de tempo.
   \item Pessoas de negócio e desenvolvedores devem trabalhar diariamente em
   conjunto por todo o projeto.
   \item Construa projetos em torno de indivíduos motivados. Dê a eles o
   ambiente e o suporte necessário e confie neles para fazer o trabalho.
   \item O método mais eficiente e eficaz de transmitir informações para e
   entre uma equipe de desenvolvimento é através de conversa face a face.
   \item Software funcionando é a medida primária de progresso.
   \item Os processos ágeis promovem desenvolvimento sustentável. Os
   patrocinadores, desenvolvedores e usuários devem ser capazes de manter um
   ritmo constante indefinidamente.
   \item Contínua atenção à excelência técnica e bom design aumenta a
   agilidade.
   \item Simplicidade -- a arte de maximizar a quantidade de trabalho não
   realizado -- é essencial.
   \item As melhores arquiteturas, requisitos e designs emergem de equipes
   auto-organizáveis.
   \item Em intervalos regulares, a equipe reflete sobre como se tornar mais
   eficaz e então refina e ajusta seu comportamento de acordo.
\end{enumerate}

Esses princípios foram a base para o que mais tarde ficou conhecido como
manifesto ágil \cite{agilemanifesto}. O manifesto ágil prega alguns valores
descritos a seguir.

\begin{quotation}
   Estamos descobrindo maneiras melhores de desenvolver software, fazendo-o nós
   mesmos e ajudando outros a fazerem o mesmo. Através deste trabalho, passamos
   a valorizar:
   
   \begin{description}
      \item[Indivíduos e interações] mais que processos e ferramentas
      \item[Software em funcionamento] mais que documentação abrangente
      \item[Colaboração com o cliente] mais que negociação de contratos
      \item[Responder a mudanças] mais que seguir um plano
   \end{description}
   
   Ou seja, mesmo havendo valor nos itens à direita, valorizamos mais os itens
   à esquerda.
\end{quotation}

Essa foi praticamente a base para todas as metodologias consideradas ágeis
atualmente. Todas tentam de maneiras diferentes fornecer os valores que foram
definidos no manifesto ágil.

\subsection{XP}

Um dos responsáveis pela revolução ágil foi Kent Beck, ao escrever o livro
Extreme Programming Explained em 1999. Ele publicou a provavelmente mais famosa
metodologia ágil até hoje, a Programação Extrema (Extreme Programming - XP)
\cite{Beck:1999:ECE:619045.621348}. Seguindo o entendimento que a sociedade
define valores para si mesma que conflitam com valores individuais, pois o que é
pensado de forma coletiva, em geral, priva as pessoas de certas liberdades. A XP
seguiu esse mesmo entendimento, que são necessários valores para indicar se o
que está sendo feito é para um bem comum ou por interesses individuais. Surgiram
então os valores da Extreme Programming.

% \textbf{Valores do XP}

\begin{enumerate}
   \item Comunicação
   \item Simplicidade
   \item Feedback
   \item Coragem
   \item Respeito
\end{enumerate}

Na primeira edição do livro citado somente os quatro primeiro valores foram
citados, só na segunda edição é que o respeito foi introduzido, pois foi
percebido que ele é um valor que está permeando os outros e é fundamental para
o sucesso do projeto \cite{beck2000extreme}.

O fato dos valores serem preservados ao longo do desenvolvimento é um bom
indicativo de sucesso. No entanto, os valores eram muito vagos e foi necessário
atacar de uma forma mais concreta, sugiram então as práticas. O XP possui um
conjunto de práticas ou princípios, que são os principais responsáveis pela
conformidade com o conceito de desenvolvimento ágil.

\textbf{Princípios ou Práticas do XP}

\begin{description}
   \item[Planejamento incremental] - Os requisitos são registrados em cartões de
   histórias e as histórias a serem incluídas em um release são determinadas
   pelo tempo disponível e sua prioridade relativa.
   \item[Pequenos releases] - O conjunto mínimo útil de funcionalidade que
   agrega valor ao negócio é desenvolvido primeiro.
   \item[Projeto Simples] - É realizado um projeto suficiente para atender aos
   requisitos atuais e nada mais.
   \item[Desenvolvimento test-fist] - Um framework automatizado de teste
   unitário é usado para escrever os testes para uma nova parte da
   funcionalidade antes que esta seja implementada.
   \item[Refactoring] - Espera-se que todos os desenvolvedores recriem o código
   continuamente tão logo os aprimoramentos do código forem encontrados.
   \item[Programação em pares] - Os desenvolvedores trabalham em pares, um
   verificando o trabalho do outro e fornecendo apoio para realizar sempre um
   bom trabalho.
   \item[Propriedade coletiva] - Os pares de desenvolvedores trabalham em todas
   as áreas do sistema, de tal maneira que não se formem ilhas de conhecimento,
   com todos os desenvolvedores de posse de todo o código.
   \item[Integração contínua] - Tão logo o trabalho em uma tarefa seja conluído,
   este é integrado ao sistema como um todo.
   \item[Ritmo sustentável] - Grandes quantidades de horas extras não são
   consideradas aceitáveis, pois, no médio prazo, há uma redução na qualidade do
   código e na produtividade.
   \item[Cliente on-site] - Um representante do usuário final do sistema (o
   cliente) deve estar disponível em tempo integral para apoiar a equipe de XP.
\end{description}

Além das práticas o XP ainda define um conjunto de atividades que servem de
orientação ao longo do desenvolvimento.

\textbf{Atividades do XP}

\begin{description}
   \item[Programar] - Ao fim do dia deve haver um programa.
   \item[Testar] - Funcionalidades de software que não podem ser demonstradas
   por testes automatizados simplesmente não exitem.
   \item[Escutar] - Os desenvolvedores devem ouvir o que os sistemas precisam
   fazer dos clientes.
   \item[Projetar] - Projetar é criar uma estrutura que organize a lógica no
   sistema.
\end{description}

\subsection{Scrum}

Por volta de 2001 Ken Schwaber escreveu o livro \textit{Agile Software
Development with Scrum}, que seria o responsável pela iniciação da metodologia
de gerenciamento de projetos ágil denominada \textit{Scrum}. O seu nome é
baseado no jogo de \textit{Rugby}, além de alguns outros conceitos. O próprio
autor define o \textit{Scrum} como um framework e não uma metodologia, uma vez
que ele não é um processo prescritivo, ou seja, não define o que deve ser feito
em todas as circunstâncias \cite{schwaber2002agile}. É fortemente embasado no
processo de controle empírico, pois acredita que tarefas complexas possuem um
grau tão auto de imprevisibilidade que é inviável tentar definir todo o
processo.

Além disso, o Scrum define três papeis. Todas as responsabilidades de
gerenciamento do projeto são divididas entre esses três papeis
\cite{schwaber2002agile}.

\textbf{Papéis do Scrum}

\begin{description}
   \item[Product Owner] - é o representante de todos os interessados no projeto
   e o sistema resultante.
   \item[Time] - é o responsável pelo desenvolvimento das funcionalidades.
   \item[ScrumMaster] - é o responsável pela correta implementação do processo
   Scrum.
\end{description}

Não existem só pessoas que estão classificadas em um dos três papeis, é normal
haver outros interessados, contudo somente as pessoas que se encaixam nos
papeis têm a total responsabilidade de fazer o necessário para o sucesso do
projeto \cite{schwaber2002agile}.

O \textit{Scrum} define ainda alguns artefatos que são utilizados durante o
gerenciamento dos projetos.

\textbf{Artefatos do Scrum}

\begin{description}
   \item[Product Backlog] - é onde estão definidos os requisitos do sistema ou o
   produto sendo desenvolvido.
   \item[Sprint Backlog] - define o trabalho, ou tarefas, selecionadas do
   product backlog a serem implementadas pelo time na próxima Sprint.
   \item[Burndown Chart] - o gráfico que mostra o conjunto de trabalho restante
   em relação ao tempo.
\end{description}

% Práticas

% \begin{description}
%    \item[Daily Scrum]
%    \item[Sprint Planning]
%    \item[Sprint Review]
%    \item[Sprint Retrospective]
% \end{description}

O Scrum segue um ciclo de desenvolvimento bem definido, conforme pode ser
visualizado na \autoref{ciclo-de-vida-scrum}.

\begin{figure}[htb]
   \caption{Ciclo de Vida do Scrum.}
   \label{ciclo-de-vida-scrum}
   \begin{center}
       \includegraphics[scale=0.4]{ciclodevidascrum.png}
   \end{center}
   \fonte{http://www.semeru.com.br/blog/category/daily-scrum-meeting/}
\end{figure}

% \subsection{Kanban}

% \subsection{Lean}

% \subsection{Anarchy Programming}

\section{MPDFT}

\label{mpdft}

O Ministério Público do Distrito Federal e Territórios (MPDFT) é um dos
integrantes do Ministério Público da União (MPU). É o responsável pela defesa
dos direitos dos cidadãos e dos interesses da sociedade. Dotado de poderes
definidos na Constituição de 1988 que garantem a independência dos três poderes
instituídos - Executivo, Legislativo e Judiciário - além da independência
funcional, a indivisibilidade e a unidade \cite{mpdft}.

O Departamento de Tecnologia da Informação (DTI) é o responsável por toda a
informatização do órgão. Várias equipes cuidam desde contratações envolvendo TI
até desenvolvimento de soluções de software passando por administração da
infraestrutura de informática (redes de computadores, banco de dados, segurança
etc).

\subsection{MPDFT-UP}

O Processo Unificado do Ministério Público do Distrito Federal e Territórios
(MPDFT-UP) \cite{mpdft-up} é o resultado das recomendações do Tribunal de Conta
da União (TCU) presentes no acórdão nº 1.603/2008 \cite{acordao-tcu-1603-2008}
e publicadas no Diário Oficial da União \cite{dou-20080818} que visam melhorias
na governança de tecnologia da informação na Administração Pública Federal. No
item 9.1.4 desse acórdão é proposta a utilização de medidas que estimulem a
adoção de metodologias de desenvolvimento de sistemas.

Sabendo que os processos evoluem para explorar as capacidades das pessoas em uma
organização e as características específicas do sistemas que estão sendo
desenvolvidos \cite{sommerville2007}, o MPDFT-UP foi desenvolvido baseando-se
no processo unificado e sendo, portanto um processo iterativo e incremental. A
utilização do processo unificado não é tão crítica em relação a necessidades e
pressões de mercado, mas, por se tratar de um órgão público, é necessária
especial atenção à documentação dos sistemas desenvolvidos, formalização de
demandas, designação de responsáveis pelo desenvolvimento e pelos aceites.

A Portaria Normativa nº 22 de 2009 é a que regula a metodologia de
desenvolvimento de sistemas de informação do MPDFT \cite{portaria-22-2009}. O
MPDFT-UP define oito etapas para o desenvolvimento de software, que são os
processos responsáveis pela organização do fluxo de trabalho.

\textbf{Etapas que compõem a metodologia de desenvolvimento de sistemas}

\begin{enumerate}
   \item Entendimento inicial
   \item Planejamento e gerência de projeto
   \item Levantamento e análise de requisitos
   \item Implementação e manutenção
   \item Testes
   \item Homologação
   \item Treinamento e implantação
   \item Avaliação
\end{enumerate}

Apesar de existirem oito etapas somente seis estão detalhadas na documentação do
MPDFT-UP \cite{mpdft-up}. A etapa de entendimento inicial e de avaliação não são
detalhadas. Na \autoref{figura-processo-mpdft-up} é possível visualizar o fluxo
de trabalho macro do processo. Além das etapas anteriores alguns papéis também
foram definidos na Tabela \autoref{tabela-papeis-mpdft-up}.

\begin{figure}[htb]
   \caption{Fluxo de trabalho do MPDFT-UP.}
   \label{figura-processo-mpdft-up}
   \begin{center}
       \includegraphics[scale=0.45]{MPDFT-UP_BFB966A8_bfb966a8_Activity.jpeg}
   \end{center}
   \fonte{\citeonline{mpdft-up}}
\end{figure}

\begin{table}[htb]
\IBGEtab{
   \caption{Papéis do MPDFT-UP.}%
   \label{tabela-papeis-mpdft-up}
}{
\begin{tabular}{| l | l |}
   \hline
   \multirow{2}{*}{Planejamento e gerência de projeto} & Líder de projeto \\
                                                       & Responsável técnico \\
   \hline
   Levantamento e análise de requisitos & Analista de negócio e requisitos de
   sistemas \\
   \hline
   \multirow{3}{*}{Implementação e manutenção} & Administrador de dados \\
                                               & Arquiteto de sistemas \\
                                               & Desenvolvedor \\
   \hline
   \multirow{2}{*}{Testes} & Analista de teste \\
                           & Executor de teste \\
   \hline
   Homologação & Usuário gestor \\
   \hline
   \multirow{2}{*}{Treinamento e implantação} & Implantador \\
                                              & Pesquisador \\
   \hline
\end{tabular}
}{
   \fonte{Produzido pelo autor.}
}
\end{table}

\section{Avaliação}

\label{avaliacao}

Este artigo tem por objetivo avaliar se a metodologia de desenvolvimento de
software do Ministério Público do Distrito Federal de Territórios está sendo
eficiente e eficaz. Os pontos a serem avaliados foram identificados através de
uma pesquisa \textit{in loco} e da observação das atividades diárias envolvendo
desenvolvimento de software.

\subsection{Problemas Identificados}

\subsubsection{Falta de Domínio do Processo}

Um dos pontos críticos identificados é justamente a falta de conhecimento
adequado sobre o processo do órgão. Antes de tecer qualquer comentário sobre as
várias situações analisadas, esta é de longe a mais significativa. A maioria dos
envolvidos sabem da existência do MPDFT-UP, mas seguer conhecem as etapas do
processo. Quase toda a aplicação do processo de desenvolvimento é feita por meio
de orientação dos integrantes mais antigos. Isso é extremamente problemático,
pois situações tratadas no fluxo do processo não são eliminados, além de serem
incluídos vícios no trabalho diário. Não é que o processo não é seguido por
falta de interesse, os integrantes das equipes de desenvolvimento não são
incentivados ao uso.

\subsubsection{Estrutura Organizacional}

A cultura, o estilo e a estrutura organizacional influenciam a maneira como os
projetos são executados \cite[pag. 30]{pmi2008pmbok}. A estrutura organizacional
do MPDFT pode ser considerada mista, mas mantém características mais funcionais
do que projetizadas. Assim, várias dificuldades das organizações matriciais
fracas ou funcionais surgem. Torna-se dificil de prever e acompanhar o ciclo de
vida do software uma vez que cada setor tem prioridades próprias. A comunicação
entre os setores é prejudicada, pois cada setor é uma caixa preta onde é muito
difícil obter informações precisas.

\subsubsection{Diferenças nos Processos}

Haver diferenças entre os processos não implica necessariamente em problemas,
ajustes podem ser feitos para melhor adequar um processo ao órgão.Conforme
citado por \citeonline{larman2007utilizando}, o PU é completamente adaptável e
aberto a boas práticas de outros metodos. A introdução do PU não visa diminuir o
valor desses outros métodos - muito pelo contrário \cite{larman2007utilizando}.
Contudo, os ajustes feitos no processo do MPDFT acabaram ofuscando alguns
procedimentos. Das nove disciplinas originais do Processo Unificado, somente
cinco são contempladas explicitamente no MPDFT-UP conforme
\autoref{tabela-disciplinas-pu-vs-mpdft-up}. Disciplinas importantíssimas são
tratadas de maneira reduzida dentro de outras etapas do processo do órgão.

\begin{table}[htb]
\IBGEtab{
   \caption{Disciplinas: Processo Unificado vs MPDFT-UP.}
   \label{tabela-disciplinas-pu-vs-mpdft-up}
}{
\begin{tabular}{| c | c |}
   \hline
   Processo Unificado \cellcolor[gray]{0.9} & MPDFT-UP \cellcolor[gray]{0.9} \\
   \hline
   Modelagem de negócios & - \\
   \hline
   - & Entendimento inicial \\
   \hline
   Requisitos & Levantamento e análise de requisitos \\
   \hline
   Análise e projeto & - \\
   \hline
   Implementação & Implementação e manutenção \\
   \hline
   Teste & Testes \\
   \hline
   - & Homologação \\
   \hline
   Implantação & Treinamento e implantação \\
   \hline
   - & Avaliação \\
   \hline
   Gestão de configuração e mudança & - \\
   \hline
   Gerência de projeto & Planejamento e gerência de projeto \\
   \hline
   Ambiente & - \\
   \hline
\end{tabular}
}{
   \fonte{Produzido pelo autor.}
}
\end{table}

A modelagem de negócio, que é uma disciplina do PU, foi em parte absorvida
pela etapa de levantamento e análise de requisitos do MPDFT-UP. A etapa de
entendimento inicial não é bem detalhada na documentação existente, mas
infere-se que esteja mais relacionada a requisitos. A etapa de implementação e
manutenção do MPDFT-UP é a que mais absorveu as atividades de análise e projeto,
contudo a análise e o projeto de software propriamente ditos não são citados
explicitamente no processo do órgão, somente o desenvolvimento da arquitetura, o
projeto do banco de dados e a prototipação são citadas. A homologação é a etapa
do processo do MPDFT que é responsável por obter a aceitação formal por parte do
usuário gestor e não possui equivalente no PU. A disciplina de gestão de
configuração e mudança é uma das mais afetadas pela estrutura de processos do
MPDFT, essa disciplina não existe diretamente no MPDFT-UP, mas em alguns
momentos é citada em outras etapas. Por exemplo, na etapa de homologação é
realizado o teste de homologação junto ao usuário gestor e caso seja encontrado
algum problema é feita uma solicitação de mudança. Logo, deve haver algum
mecanismo previsto para gerenciar as solicitações de mudança. Por fim, a
disciplina de ambiente é outra que no dia a dia do órgão não é tão problemática,
mas para a área de desenvolvimento é bastante negligenciada. Há pouca
padronização de ferramentas, frameworks e APIs utilizadas. Essas poucas não
estão corretamente documentadas, estão guardadas nas cabeças de algumas pessoas,
o que dificulta a disseminação do conhecimento sobre o que deve ser utilizado em
 cada situação.

\subsubsection{Dificuldades nas Abordagens Ágeis}

Há ainda algumas tentativas tímidas, porém com enorme potencial, de abordagens
agilistas em meio aos processos de desenvolvimento de software do órgão. Há
algumas equipes utilizando, de maneira experimental, Scrum e Kanban. Além de
tentativas de adotar práticas do XP como a integração contínua. Contudo, novos
problemas trazem novas dificuldades e uma das mais críticas em relação às
abordagens ágeis é a falta de comprometimento do cliente. Conforme explicado
anteriormente, um dos fatores cruciais para o sucesso de metodologias ágeis é o
envolvimento do cliente e isso não é, em regra, respeitado no MPDFT. No
Ministério Público do Distrito Federal e Territórios a atividade fim está
relacionada à fiscalização da aplicação da Lei. Logo, os membros, oriundos da
área de direito, são o topo da cadeia. Eles nem sempre estão interessados nas
técnicas, tecnologias, metodologias, e etapas necessárias para a construção de
um sistema. Portanto, é muito difícil contar com a disponibilidade do maior
interessado no trabalho final. Isso por si só já é um fator que pesa bastante na
adoção de várias práticas agilista, mas o envolvimento do cliente é crucial para
qualquer abordagem e não é por isso que o desenvolvimento de software deve
parar. Como relatado anteriormente, a estrutura organizacional, que influencia
na organização das equipes, também dificulta abordagens ágeis mais eficientes
uma vez que o modelo adotado é mais voltado para uma organização funcional do
que projetizada. \textcolor{red}{VERIFICAR FONTE PARA ARGUMENTAR QUE
METODOLOGIAS ÁGEIS TENDEM A SER MAIS PROJETIZADAS.}

\subsection{Sugestões e Melhorias}

\subsubsection{Capacitação do Integrantes}

Já que existe um processo bem definido e uma fase de integração ou ambientação
dos novos integrantes do órgão, uma das ações tomadas pode ser a inclusão de
treinamentos não só no processo, mas nas tecnologias que serão utilizadas,
sistemas a serem mantidos etc. A inclusão de um treinamento é fundamental para a
posterior cobrança por parte dos membros das equipes de desenvolvimento.

\subsubsection{Organização Projetizada}

Ao se escolher o tipo de estrutura organizacional se está, na verdade,
priorizando o que será mantido. Uma estrutura funcional prioriza os produtos, já
uma estrutura projetizadas prioriza o desenvolvimento de projetos
\cite{larson:1999:IEEE}. Como há muitos projetos novos sendo desenvolvidos e
outros esperando na fila para desenvolvimento futuro, priorizar o andamento dos
projetos existentes é fundamental. A diminuição de gargalos e melhorias no
processo passam necessariamente por uma reestruturação das equipes e da forma de
trabalho atual.

\subsubsection{Melhorias no Processo}

Primeiro, a modelagem de negócio é uma disciplina que pode ser muito bem
empregada no entendimento do negócio atual do cliente, que é a área fim do
MPDFT.

A disciplina de análise e projeto realmente pode continuar fazendo parte da
implementação, principalmente, com a adoção mais agressiva de metodologias
ágeis.

Deve haver uma sintonia maior entre as áreas de implementação e teste, o
trabalho entre os responsáveis pelas duas disciplinas pode acontecer de uma
forma mais adequada para a produtividade. A implementação pode ter como saída
código coberto por testes e os responsáveis pela disciplina de teste pode
fornecer o roteiro de teste das funcionalidades desenvolvidas. Assim, as duas
áreas ganham, a implementação por conseguir cobrir cenários não planejados e
enviar uma funcionalidade mais robusta para os responsáveis pelo teste e o teste
por receber algo mais maduro e, eventualmente, até com testes automatizados.

Treinamento e implantação, principalmente em se tratando de implantação, há
pouca documentação sobre os ambientes de desenvolvimento, teste, homologação e
produção. Pouca documentação sobre passos a serem seguidos para publicar
corretamente uma aplicação em outro ambiente, formato de e-mail de liberação de
versaão etc.

As disciplinas de gestão de configuração e mudança e ambiente, que não estão
presentes no MPDFT-UP deveriam ganhar mais espaço, principalmente, por serem
disciplinas de apoio, ou seja, trabalham em prol de todo o processo. Uma
política de gestão de configuração minimizaria o tempo de aprendizado de um novo
integrante, já que ele não vai precisar aprender a forma que uma nova equipe faz
controle de código. A padronização do que é armazenado em controle de versão e
estrutura de diretórios também é importante. Para ambiente, é bom estar escrito
qual versão das ferramentas são utilizadas no órgão.

Limites bem definidos entre os modelos de processos é algo muito bom do ponto de
vista didático, mas, no mundo prático, é inviável se utilizar somente de um
modelo de desenvolvimento de software. Conforme explicado anteriormente, os
modelos possuem vantagens e desvantagens, o que os tornam interessantes sob
determinadas situações. Contudo, instituições sólidas buscam uma padronização no
modo de trabalho, pois um modo de trabalho uniforme viabiliza o aprimoramento
dos processso de software, no qual a diversidade de processos de software ao
longo da organização é reduzida \cite{sommerville2007}. Isso visa melhorias na
comunicação e redução na curva de aprendizado de novos colaboradores.

Tendo em vista que o Processo Unificado é bastante flexível e aberto é
incentivada a inclusão de práticas interessantes de outros métodos iterativos
tais como: eXtreming Programming (XP) \cite{Beck:1999:ECE:619045.621348}, Scrum
\cite{schwaber2002agile}. Assim, é possível obter vantagem dos dois mundos.

\subsubsection{Revisões Periódicas}

Outra melhoria que pode ser feita no processo é a revisão periódica do próprio
processo, das tecnologias e das práticas adotadas. Muito do que é utilizado
atualmente no MPDFT é defasado tecnologicamente. Tanto em termos procedimentais
quanto tecnológicos.

% Finaliza a parte no bookmark do PDF, para que se inicie o bookmark na raiz
% ---
\bookmarksetup{startatroot}% 
% ---

% ---
% Conclusão
% ---
\section*{Considerações finais}
\addcontentsline{toc}{section}{Considerações finais}

Grande parte dos problemas relacionados à utilização do Processo Unificado como
base para o desenvolvimento/manutenção de software se deve a tentativa de tornar
um processo já existente em nível executado e às vezes até definido em outro
processo. O que geralmente, e erroneamente, leva a equipe da qualidade a forçar
a utilização de um novo e teoricamente melhor modelo. Assim, em vez de analisar
 como o software é tratado em uma organização e simplesmente documentar, a
equipe de qualidade tenta sair de algo executado de forma meramente empírica, em
muitos casos, para algo altamente documentado e, muitas vezes, burocrático. Essa
é uma grande queixa por parte dos envolvidos no uso do processo e um dos
principais responsáveis por um dos grandes mitos envolvendo o PU, o processo é
pesado e, portanto, burocrático.

Conforme explicado nas seções anteriores, não há certos ou errados no mundo de
desenvolvimento de software. É tudo uma questão de adaptação das
técnicas/práticas no dia a dia. É completamente possível utilizar o Processo
Unificado de forma ágil e tornar uma metodologia ágil lenta e burocrática.
Portanto, não é necessário mudar completamente o processo de desenvolvimento de
software do MPDFT, basta adaptá-lo às reais necessidades do órgão.

Ao avaliar os problemas e as sugestões de melhoria será possível evoluir a forma
de trabalho atual e obter grande retorno por parte dos participantes e do
próprio processo. Mas, as melhorias não devem parar com este artigo novos
problemas surgirão e novas soluções serão propostas, pois um processo de
desenvolvimento de software trata de processos, pessoas e ferramentas, logo, é
vivo e deve estar em constante evolução.

% ----------------------------------------------------------
% ELEMENTOS PÓS-TEXTUAIS
% ----------------------------------------------------------
\postextual

% ----------------------------------------------------------
% Referências bibliográficas
% ----------------------------------------------------------
\bibliography{abntex2-modelo-references}

\end{document}
